\documentclass{ltjsarticle}
% 数式
\usepackage{amsmath}
% 画像表示
\usepackage{graphicx}
\usepackage{svg}
% 表を使いやすく
\usepackage{tabularx}
% 図の場所を強制
\usepackage{here}
% URL
\usepackage{url}
% ハイパーリンクがつく
\usepackage{hyperref}
% 文章を枠で囲う
\usepackage{ascmac}
% enumerateのラベルを変更可能に
\usepackage{enumitem}
% 外部のファイルからテキストを読み込む
\usepackage{moreverb}
% ソースコードの表示用
\usepackage{listings}
%ソースコードの表示に関する設定
\lstset{
    basicstyle={\ttfamily},
    identifierstyle={\small},
    commentstyle={\smallitshape},
    keywordstyle={\small\bfseries},
    ndkeywordstyle={\small},
    stringstyle={\small\ttfamily},
    frame={tb},
    breaklines=true,
    columns=[l]{fullflexible},
    numbers=left,
    xrightmargin=0\zw,
    xleftmargin=3\zw,
    numberstyle={\scriptsize},
    stepnumber=1,
    numbersep=1\zw,
    lineskip=-0.5ex
}
\renewcommand{\lstlistingname}{リスト}
% 中央揃え(tabularx)
\newcolumntype{C}{>{\centering\arraybackslash}X}
% 左揃え(tabularx)
\newcolumntype{L}{>{\scriptsize\raggedright\arraybackslash}X}
% 右揃え(tabularx)
\newcolumntype{R}{>{\scriptsize\raggedleft\arraybackslash}X}
% 行の中心に配置
\newcommand{\CenterRow}[2]{
    \dimen0=\ht\strutbox%
    \advance\dimen0\dp\strutbox%
    \multiply\dimen0 by#1%
    \divide\dimen0 by2%
    \advance\dimen0 by-.5\normalbaselineskip%
    \raisebox{-\dimen0}[0pt][0pt]{#2}
}


\title{\vspace{-4cm}
制御工学実験Ⅱ テーマ:マイコン基礎応用
}
\author{
    制御情報システム工学科 4年 10番 國安柾希
}

\begin{document}
\maketitle

\subsection*{実施評価}
\begin{tabularx}{\textwidth}{|p{130mm}|C|C|} \hline
    \CenterRow{2}{評価項目} & 自己評価 & 担当評価 \\ \hline
    実験開始までに実験テキストや実験ノートを準備できており,事前課題がある場合は,それに取り組んでいた.
    & \CenterRow{2}{A} & \\ \hline
    担当者による指示をよく聞き,不注意による無用な誤りなく安全に実験を行うことができた.
    & \CenterRow{2}{A} & \\ \hline
    回路やプログラムを自分で作成し,グループワークの場合は自らの役割を全うするなど,課題に対して積極的に取り組むことができた.
    & \CenterRow{2}{A} & \\ \hline
    与えられた課題を時間内に達成し,結果を正確に記録または出力できた. \par
    & \CenterRow{2}{A} & \\ \hline
    使用器具の後片付けや実験場所の清掃をきちんと行った. \par
    & \CenterRow{2}{A} & \\ \hline
\end{tabularx}

\subsection*{レポート評価}
\begin{tabularx}{\textwidth}{|p{130mm}|C|C|} \hline
    \CenterRow{2}{評価項目} & 自己評価 & 担当評価 \\ \hline
    章立ては適切であり,それぞれの章における記載内容は自作のものである.引用がある場合は,その旨を明記している.
    & \CenterRow{2}{A} & \\ \hline
    図・表の書き方は裏面の要領に準じており,自作のものである.(担当者が許可しない限り,指導書の図すら引用してはいけない)
    & \CenterRow{2}{A} & \\ \hline
    使用器具や実験環境について,実験結果を再現するのに十分な情報を記載している. \par
    & \CenterRow{2}{A} & \\ \hline
    課題に関する計測結果や出力結果を整理して記載し,結果に対する独自の考察を述べている.
    & \CenterRow{2}{A} & \\ \hline
    研究課題に取り組み,適切な参考文献を基に答えを導き出している. \par
    & \CenterRow{2}{A} & \\ \hline
\end{tabularx}

\begin{center}
    \begin{tabularx}{100mm}{|C|C|C|} \hline
        実施点\par(50) & レポート点\par(50) & 合計点\par(100) \\ \hline
        \ \par\ \par & & \\ \hline
    \end{tabularx}
\end{center}

\newpage

\section{実験目的}

今までの実験では,PCで作成したプログラムをArduinoに書き込んで実行してきた.しかし,Arduinoへの書き込みのみでは,要求する動作が複数ある場合,選択することが困難になる場合がある.
それを解決するために,今回の実験では,Arduinoのプログラムを書き込んだ後にArduinoとPC間でシリアル通信を行って,PCからコマンドを送信してハードウエアの制御を行う.
また,Webサイトから必要な情報を収集し実際に実装することで,具体的な知識やスキルを身に着け,そのシステム構築を通して,PCとArduinoの連携を利用した,より複雑なシステムの構築方法を身に着ける.

% タイトルは自分で考える
% 「実験目的」で出したキーワードについて,数式やフローチャート,図や表を利用して詳細な仕組みを説明する.

\section{構築した組み込みシステム}
\subsection{実験課題}
以下の条件を満たすシステムを構築する
% 半固定抵抗のつまみの角度(位置)に対応してサーボモータの
% 角度を変える(自動車のパワーステアリングのイメージ)
\begin{itemize}
    \item 入力:シリアル通信(Rx),半固定抵抗(10kΩ)
    \item 出力:シリアル通信(Tx),赤黄LEDを各1個,サーボモータ
    \item 動作:\begin{enumerate}
        \item PCからコマンド操作(シリアルモニタから文字列を送信して操作)
        “r,ooo”: 赤色LEDをoooの明るさにする(oooは0-255の整数)
        “y,ooo”: 黄色LEDをoooの明るさにする
        “w,ooo,OOO”: 赤色LEDをooo,黄色LEDをOOOの明るさにする
        “s”: 赤色LEDと黄色LEDの現在の明るさをPCに送る
        \item 半固定抵抗のつまみの角度(位置)に対応してサーボモータの角度を変える(自動車のパワーステアリングのイメージ)
    \end{enumerate}
\end{itemize}

\subsection{実験結果}
作成したシステムのプログラム,回路図を以下に示す.
\newpage
\begin{lstlisting}[caption=,label=]
  
  #include <Servo.h> //ライブラリ<Servo.h>を組み込む
  #define RPIN 5
  #define YPIN 3
  #define RLPIN A1
  #define YLPIN A2
  
  int value;
  
  Servo sv; //Servoオブジェクト“sv”を作成する
  
  void setup() {
   Serial.begin(9600); //シリアル通信のデータ転送レートを9600ビット/秒に設定する
   // Servo
   //svの出力をリセットした上で、D6番ピンに割り当て、パルス幅を500~2400マイクロ秒とする
   sv.attach(6, 500, 2400);
   
   // led
   pinMode(RPIN, OUTPUT);
   digitalWrite(RPIN, LOW);
   pinMode(YPIN, OUTPUT);
   digitalWrite(YPIN, LOW);
  }
  
  void loop() {
    // servo
    int angle = analogRead(0) ; //変数angleをA0番ピンの入力値とする
    //angleを500~2400の整数値に変換してD6番端子に出力する
    sv.writeMicroseconds(map(angle, 0, 1023, 500, 2400)) ;
    delay(20) ;
   
    //  led
    while (Serial.available()) {
      String str = Serial.readString();//cmd = str
      char ini = str[0];
  
      if(ini == 'r'||ini == 'y'){
        int value = str.substring(2).toInt();
        Serial.println(value);
        if(ini == 'r'){
          analogWrite(RPIN,value);
          delay(20);
        }else{
          analogWrite(YPIN,value);
          delay(20);
        }
      }else if(ini == 'w'){
        int Comma = str.indexOf(",", 2);
        int value_r = str.substring(2, indexComma2).toInt();
        int value_y = str.substring(indexComma2+1).toInt();
        Serial.println(value_r);
        Serial.println(value_y);
        analogWrite(RPIN, value_r);
        analogWrite(YPIN, value_y);
      }else if(ini == 's'){
        int value_r = map(analogRead(RLPIN),0,1023,0,255);
        int value_y = map(analogRead(YLPIN),0,1023,0,255);
        Serial.println("==== status ====");
        Serial.println("RED LED : ");
        Serial.println(value_r);
        Serial.println("YELLOW LED : ");
        Serial.println(value_y);
      }
      
    }
  }
    
\end{lstlisting}

\section{考察}




\section{感想}

\begin{thebibliography}{9}
    % \bibitem{key} title, hostname, \url{url}, (参照日 YYYY年 MM月 DD日)
\end{thebibliography}
\end{document}
